% <!--
%   CE FICHIER CONTIENT LES DÉFINITIONS GÉNÉRALES DES LIENS
% 
%   DÉFINITION :
% 
%   [un texte identifiant]:  une/path/to/cible  "Le titre optionnel"
% 
%   UTILISATION
% 
%   avec le texte identique :
% 
% un texte identifiant][
% 
%   avec un autre texte :
% 
% autre texte pour le lien][un texte identifiant
% 
%   -->

\section{Les types de travaux}\hypertarget{type-travaux}{}\label{type-travaux}

Les “travaux” du programme \unan{} sont classés en trois grands types~{}:

\begin{description}
\item[Les lectures de pages de cours] \hfill \\
 Ces travaux concernent principalement votre apprentissage de la narration.



\item[Les questionnaires et autres quiz] \hfill \\
 Ils permettent notamment de valider vos acquis au fil du programme, ou de procéder à des checks réguliers de votre projet.



\item[Les actions forum] \hfill \\
 qui concerne le Forum du site.



\item[Les actions ou tâches en tout genre] \hfill \\
 Ce sont les travaux autres que les travaux précédents, à commencer par les plus importants~{}: la rédaction des documents de travail de votre projet.
\end{description}

\subsection{Travaux sur votre bureau}\hypertarget{travaux-sur-votre-bureau}{}\label{travaux-sur-votre-bureau}

Sur votre \hyperlink{bureau}{bureau de travail}, les travaux sont répartis en plusieurs onglets distincts qui reprennent ces grands types de travaux~{}: l'onglet ``Tâches'', l'onglet ``Cours'', l'onglet ``Forum'' et l'onglet ``Quiz''.

Dès qu'un type contient des travaux, le nombre de ces travaux est indiqué dans l'onglet lui-même, après le nom.

