\begin{itemize}
\item \href{manuel/rediger?in=analyse\&manp=analyse/analyse\_mye\#presentationdesanalysesmye}{Présentation des analyses MYE}
\item \href{manuel/rediger?in=analyse\&manp=analyse/analyse\_mye\#realisationfichiertdm}{Réalisation d'un fichier de table des matières}
\item \href{manuel/rediger?in=analyse\&manp=analyse/analyse\_mye\#fichierintroductionauto}{Fichier d'introduction de l'analyse}
\end{itemize}



\section{Présentation des analyses MYE}\hypertarget{prsentation-des-analyses-mye}{}\label{prsentation-des-analyses-mye}

Les “Analyses MYE” sont des analyses fonctionnent sur la base de trois types de fichier~{}:

\begin{itemize}
\item \textbf{markdown}. Pour écrire toute sorte d'articles,
\item \textbf{yaml}. Pour réaliser toute sorte de listings (procédés, notes, etc.),
\item \textbf{evc}. Format propre pour les évènemenciers.
\end{itemize}



\subsubsection{Réalisation d'un fichier de table des matières}\hypertarget{ralisation-dun-fichier-de-table-des-matires}{}\label{ralisation-dun-fichier-de-table-des-matires}

Pour réaliser un fichier de table des matières, il faut créer le fichier~{}:

\begin{verbatim}tdm.yaml
\end{verbatim}

… à la racine du dossier de l'analyse.

Dans ce fichier, on appelle chaque fichier à l'aide d'une définition~{}:

\begin{verbatim}-
  path:     path/to/file.ext
  titre:    Titre à donner dans le fichier de l'analyse
\end{verbatim}

Le \texttt{path} se calcule depuis la racine du dossier de l'analyse du film concerné.

D'autres éléments peuvent être définis comme~{}:

\begin{verbatim}-
  ...
  
  incipit:      Un texte qui sera ajouté en haut du fichier, en 
                italic et en plus petit.
  conclusion:   Un texte qui sera ajouté à la fin du fichier.
\end{verbatim}

\begin{quote}
Note~{}: \texttt{incipit} et \texttt{conclusion} sont utiles par exemple pour les fichiers YAML et EVC qui ne peuvent pas contenir de texte hors de leurs rubriques normales.
\end{quote}



\section{Fichier d'introduction de l'analyse}\hypertarget{fichier-dintroduction-de-lanalyse}{}\label{fichier-dintroduction-de-lanalyse}

S'il existe un fichier à la racine du dossier d'analyse qui s'appelle~{}:

\begin{verbatim}introduction.tm
\end{verbatim}

… alors il est automatiquement chargé en début d'analyse.

