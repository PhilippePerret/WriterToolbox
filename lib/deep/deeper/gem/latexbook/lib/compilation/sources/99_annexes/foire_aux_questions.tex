% <!--
%   CE FICHIER CONTIENT LES DÉFINITIONS GÉNÉRALES DES LIENS
% 
%   DÉFINITION :
% 
%   [un texte identifiant]:  une/path/to/cible  "Le titre optionnel"
% 
%   UTILISATION
% 
%   avec le texte identique :
% 
% un texte identifiant][
% 
%   avec un autre texte :
% 
% autre texte pour le lien][un texte identifiant
% 
%   -->

\section{Foire aux questions}\hypertarget{faq}{}\label{faq}

\begin{description}
\item[Est-ce que l'inscription au site vaut inscription au programme~{}?] \hfill \\
 Non, l'inscription au site \boa{} et l'inscription au programme \unan{} sont deux choses bien distinctes.



De la même manière, l'abonnement au site permet d'utiliser l'intégralité des outils mais ne permet pas de suivre le programme.



En d'autres termes, le programme \unan{} est un outil particulier et tout à fait à part sur le site.



\item[Puis-je mettre mon programme en pause~{}?] \hfill \\
 Pour le moment, la procédure n'est pas possible, mais si vous insistez auprès de l'administration, un arrangement devrait toujours être possible ;-).



\item[Puis-je soumettre mon travail à quelqu'un~{}?] \hfill \\
 Oui, tout à fait, et c'est même conseillé, en utilisant \leForum{} de l'atelier.



\item[Peut-on contacter d'autres auteurs du programme~{}?] \hfill \\
 Tout à fait~{}! À partir du moment où ils ont réglé leurs \hyperlink{preferences-auteur}{préférences} pour permettre d'être contacté par d'autres auteurs du programme.



\item[Où peut-on joindre Philippe Perret~{}?] \hfill \\
 Vous pouvez le joindre en utilisant son mail~{}: \mailphil{phil@laboiteaoutilsdelauteur.fr} ou en passant par l'\mailadmin{administration}.
\end{description}

