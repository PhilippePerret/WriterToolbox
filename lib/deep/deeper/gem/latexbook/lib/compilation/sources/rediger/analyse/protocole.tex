Ce fichier présente le protocole d'analyse de film adopté sur La Boite à Outils de l'Auteur.

\subsection{1. Collecte}\hypertarget{collecte}{}\label{collecte}

Collecte des informations générales du film pour réalisation de l'entête du fichier et création de son entrée dans le  (pour cette opération, le grade administrateur est requis —~{}si vous ne possédez pas ce grade, vous pouvez demander à un administrateur de créer la fiche pour vous).

Collecte systématique des informations du film au format de fichier \texttt{.film} propre aux analyses de La Boite à Outils de l'Auteur (pour ce faire, vous pouvez utiliser le \href{http://www.github.com/philippeperret}{bundle TextMate d'analyse de film} ou utiliser l’\href{analyse/collecteur}{outil de collecte en ligne}).

Cette collecte se fait le plus rigoureusement possible, en visionnant le film scène après scène, plan après plan, en relevant chaque information, chaque procédés, chaque conflit, etc. rencontré, de la façon la plus objective possible.

\subsection{2. Analyse objective de la collecte}\hypertarget{analyse-objective-de-la-collecte}{}\label{analyse-objective-de-la-collecte}

Une fois la collecte rigoureuse opérée, on peut produire les premiers résultats statistiques qui permettent de tirer des premières conclusions.

Les premiers fichiers d'analyse sont donc des fichiers qui s'appuie intégralement sur les résultats statistiques et en tirent les conséquences qui s'imposent.

C'est dans cette partie notamment qu'on pourra déterminer le \href{scenodico/8/show}{protagoniste} en fonction des temps de présence des personnages, qu'on pourra déterminer le plan adopté par l'auteur, etc.

Un autre fichier pourra comparer les résultats statistiques du film avec les résultats statistiques des autres films déjà analysés.

\subsection{3. Analyse subjective}\hypertarget{analyse-subjective}{}\label{analyse-subjective}

Fort de tous ces éléments, l'analyste possédant un grade suffisant pourra rédiger des fichiers d'analyse plus subjectifs s'appuyant sur les données collectées et les conclusions tirées ainsi que sur ses propres impressions.

Ces fichiers pourront mettre en exergue un élément qui n'apparait pas, ou trop peu, dans l'analyse objective, ou souligner des points remarquables du film.

\subsection{4. La leçon tirée du film}\hypertarget{la-leon-tire-du-film}{}\label{la-leon-tire-du-film}

Dans un dernier temps, il s'agira de déterminer la \emph{leçon d'écriture} qu'on peut tirer du film et de l'expliciter en détail, le plus précisément possible.

Cette “leçon” s'appuie sur un point particulièrement remarquable du film qui permet donc de le prendre comme illustration d'un procédé d'écriture particulier.

Cette leçon n'est pas évidente à déterminer pour un apprenti-analyste ou un apprenti-auteur, c'est la raison pour laquelle elle n'est pas obligatoire pour les grades inférieurs.

[
Noter que suivant son grade, et particulièrement pour les grades inférieurs, l'analyste est accompagné tout au long de son analyse, une aide lui est apportée ainsi qu'un regard critique sur les documents qu'il produit, qui lui permettent d'approfondir de façon dirigée son travail d'analyse.

En d'autres termes, un analyste qui débute sur La Boite à Outils de l'Auteur se retrouve accompagné par des analystes plus chevronnés déjà présents. Aucune crainte à avoir~{}!
]

