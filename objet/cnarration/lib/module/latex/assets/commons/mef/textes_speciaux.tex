% Définition des styles propres à la série Narration
% Définition des textes spéciaux
%
% Note ajout
%   Quand des styles sont ajoutés, ils peuvent être aussi ajoutés à la liste
%   du fichier ./scripts/data/applescript/style_list.txt pour pouvoir être
%   proposés par la macro AS de TexShop (command + Control + S)

% ---------------------------------------------------------------------
%   Définition des termes spéciaux
% ---------------------------------------------------------------------

% Terme technique non indexé : \tterm{...}
\newcommand{\tterm}[1]{\texttt{#1}} 

% Film dans le texte
\newcommand{\film}[1]{
  \small \textit{\MakeTextUppercase{#1}}
}

% Personnage dans le texte
\newcommand{\personnage}[1]{#1}
% Réalisateur dans le texte
\newcommand{\realisateur}[1]{#1}
% Auteur dans le texte
\newcommand{\auteur}[1]{#1}
% Acteur dans le texte
\newcommand{\acteur}[1]{#1}

% Un volume de la collection cité dans le texte
\newcommand{\bookserie}[1]{%
  livre \enquote{#1} de la série%
}
\newcommand{\volumeserie}[1]{%
  volume \enquote{#1}%
}

% Mettre un texte en \exergue
% --------------------------
% TODO: Pour le moment, je le mets simplement en gras, mais il faudra dans
% l'avenir le mettre dans un cadre, un caractère plus gros peut-être, etc.
\newcommand{\exergue}[1]{
  \textbf{#1}
}