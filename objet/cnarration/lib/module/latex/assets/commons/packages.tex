%
% Packages communs à tous les livres
%

% Packages
\usepackage[french]{babel}
\usepackage[a5paper]{geometry}
\usepackage[utf8]{inputenc}
% \usepackage{fourier}
\usepackage[OT1,T1]{fontenc}
% \usepackage[cyr]{aeguill}
\usepackage[dvips]{graphicx}
\usepackage{epsfig}
\usepackage{float}
\usepackage{xcolor}
\usepackage{makeidx}
\usepackage{fancyhdr}

% == Pour les textes ==
\usepackage{textcase} % Pour \MakeTextUppercase

\usepackage[normalem]{ulem}
\usepackage{array} % Pour les tableaux plus complexes ?
% \usepackage{tabularx} % Pour les tableaux plus complexes aussi

% Pour pouvoir utiliser des citations bibliographiques plus précises, comme
% "\citet" pour écrire le titre
% \usepackage{natbib}
% biblatex plutôt que bibtex ou natbib
%\usepackage[style=apa, natbib]{biblatex}
% \usepackage[backend=biber, bibencoding=utf8, natbib]{biblatex}

% Pour les guillemets français
% \usepackage[babel=true]{csquotes}

% Pour les références croisées entre documents
\usepackage{xr}
% Pour pouvoir définir la mise en forme des références, par exemple un
% "p." avant le numéro de page.
\usepackage{varioref}
