\pagestyle{fancy}

% Effacer tout dans le pied de page et l'entête
\fancyhead{}
\fancyfoot{}

% = Haut de page =
\fancyhead[LE,RO]{\bfseries \leftmark} % Avec capitales
\renewcommand{\headrulewidth}{0.4pt} % Filet
% = Pied de page =
\fancyfoot[RE,LO]{\nouppercase{\small\rightmark}}
\fancyfoot[LE,RO]{\thepage}
\renewcommand{\footrulewidth}{0.4pt} % Filet


% Pour que les pages de chapitre soient bien formatées
% Le style est le même que pour les autres pages, sauf l'entête qui
% reste vide.
\fancypagestyle{plain}{%
  \fancyhead{}
  \renewcommand{\headrulewidth}{0pt} % Pas de filet
  \fancyfoot[RE,LO]{\nouppercase{\small \rightmark}}
  \fancyfoot[LE,RO]{\thepage}
  \renewcommand{\footrulewidth}{0.4pt}%
}

% Redéfinition de l'apparence des chapitres
\makeatletter
% Pour définir la ligne en dessous et au-dessus (je ne l'utilise pas)
% \def\thickhrule{\leavevmode \leaders \hrule height 1ex \hfill \kern \z@}
% Pour définir \position
% \def\position{\centering}
\renewcommand{\@makechapterhead}[1]{%
  \vspace*{10\p@}%
  % {\parindent \z@ \position \reset@font
  {\parindent \z@ \reset@font
        % % Le numéro du chapitre
        % {\Huge \scshape  \thechapter }
        % \par\nobreak
        % \vspace*{10\p@}%
        % \interlinepenalty\@M
        % \thickhrule
        % \par\nobreak
        % \vspace*{2\p@}%
        {\Huge \bfseries #1\par\nobreak}
        \par\nobreak
        % \vspace*{2\p@}%
        % \thickhrule
    \vskip 40\p@
    % \vskip 100\p@
  }
}

% \makeatletter
\renewcommand{\@makesectionhead}[1]{%
  {\bfseries #1\par\nobreau}
}
