% Mise en forme des tableaux, afin de garder une cohérence visuelle

% Définition d'un nouveau type de colonne, "M" pour un long texte
% aligné à gauche dans une cellule
% Nécessite le package `array'
% \newcolumntype{M}[1]{>{\raggedright}m{#1}}

% Un tableau de type exemples, ou la colonne gauche est plus petite que
% la colonne droite (rapport de 1/3)
% Notes
% -----
%   * Le "*" après tabular est la variante qui permet de définir la taille du tableau
%   * Le format de colonne "M" est défini ci-dessus
% 
% \newenvironment{tableauExemple}
%   {\small %
%   \begin{tabular*}{0.75\textwidth}{ M | M }%
% }
% {%
%   \normalsize \end{tabular*}%
% }
\newenvironment{tableauExemple2cols}
{\begin{center} \small \vspace{10pt} \begin{tabular*}{1.0\textwidth}{p{0.2\textwidth} | p{0.7\textwidth}}\hline}
{\\ \hline \normalsize \end{tabular*} \end{center} \vspace{10pt}}