%
% Packages communs à tous les livres
%

% Packages
\usepackage[french]{babel}

% Pour demander à frenchb de ne pas s'occuper des listes à
% puce.
\frenchbsetup{StandardLists=true}


\usepackage[a5paper]{geometry}
\usepackage[utf8]{inputenc}
\usepackage{fourier}
\usepackage[OT1,T1]{fontenc}
\usepackage[cyr]{aeguill}
\usepackage[dvips]{graphicx}
\usepackage{epsfig}
\usepackage{float}
\usepackage{xcolor}
\usepackage{makeidx}
\usepackage{fancyhdr}

% Pour les commandes dynamiques
% avec \DeclareDocumentCommand (permet de faire des commandes
% avec des arguments optionnels IfNoValue)
\usepackage{xparse}

% Pour pouvoir utiliser \setstrech qui définit le line-height
\usepackage{setspace}

% Pour les hyperliens dans le texte
\usepackage{hyperref}
\hypersetup{
colorlinks=false, % Pour ne pas mettre les liens en couleur
linkcolor=blue,
}
% Avec :
% \usepackage[dvips]{hyperref}
% … une erreur est produite disant : Wrong DVI mode driver option `dvips',
% because pdfTeX or LuaTeX is running in PDF mode.

% == Pour les textes ==
\usepackage{textcase} % Pour \MakeTextUppercase

\usepackage[normalem]{ulem}
\usepackage{array} % Pour les tableaux plus complexes ?
% \usepackage{tabularx} % Pour les tableaux plus complexes aussi

% Pour pouvoir utiliser des citations bibliographiques plus précises, comme
% "\citet" pour écrire le titre
% \usepackage{natbib}
% biblatex plutôt que bibtex ou natbib

% Le style 'apa' ne fonctionne pas avec les données
% que j'ai
% \usepackage[style=apa, natbib]{biblatex}

\usepackage[bibencoding=utf8, natbib]{biblatex}

% \usepackage[backend=biber, bibencoding=utf8, natbib]{biblatex}
% Définition des bibliographies pour biblatex
% AVANT mais déprécié :
% \bibliography{../commons/filmography,../commons/bibliography}
% Nouveau format :
\addbibresource{../commons/filmography.bib}
\addbibresource{../commons/bibliography.bib}

% Pour les guillemets français
\usepackage[babel=true]{csquotes}

% Pour les références croisées entre documents
\usepackage{xr}
% Pour pouvoir définir la mise en forme des références, par exemple un
% "p." avant le numéro de page.
\usepackage{varioref}

% Kramdown transforme certains truc en longtable, c'est-à-dire
% en environnement permettant de faire des tableaux sur plusieurs
% page. Il faut le package longtable pour faire ça
\usepackage{longtable}



% Pour pouvoir utiliser la font Zapf dingbats qui permet de
% modifier les puces des listes
\usepackage{pifont}
