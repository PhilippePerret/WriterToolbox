% <!--
%   CE FICHIER CONTIENT LES DÉFINITIONS GÉNÉRALES DES LIENS
% 
%   DÉFINITION :
% 
%   [un texte identifiant]:  une/path/to/cible  "Le titre optionnel"
% 
%   UTILISATION
% 
%   avec le texte identique :
% 
% un texte identifiant][
% 
%   avec un autre texte :
% 
% autre texte pour le lien][un texte identifiant
% 
%   -->

\section{Jour-programme}\hypertarget{explicationjourprogrammejourreel}{}\label{explicationjourprogrammejourreel}

Quand on suit le programme \unan{}, il faut comprendre qu'il y a deux sortes de jours~{}: les ``jours-programme'' et les ``jours réels''. Lorsque l'on suit le programme en rythme \emph{normal}, c'est-à-dire \emph{moyen}, un jour-programme et un jour réel sont équivalent ou, pour le dire autrement, un jour-programme dure la durée un jour réel, c'est-à-dire \emph{24 heures}.

Le programme \unan{} étant pensé sur un an, il est décomposé en \textbf{366 jours-programme} qui tous ensemble définissent le programme complet.

Lorsque l'on ralentit son rythme de travail, les jours réels sont plus nombreux —~{}on peut faire le programme \unan{} en un an et demi par exemple~{}— mais le nombre de jours-programme, lui, ne varie pas, il y en aura toujours 366. C'est leur durée qui se modifie, s'allongeant lorsque le rythme diminue pour donner plus de temps pour réaliser les travaux demandés.

Si, par exemple, vous suivez le programme \unan{} avec un rythme deux fois plus lent que le rythme moyen, vous mettrez deux ans pour suivre l'intégralité du programme (\emph{cela peut vous sembler long, en réalité, il faut bien plus de temps que ça à un auteur pour apprendre la dramaturgie et réussir son premier bon scénario~{}! En tout cas avant le program \textbackslash{}unan\{}\}). Deux ans pour suivre le programme prévu sur une année, cela signifie qu'un \emph{jour-programme} durera en temps \textbf{2 jours réels}.

