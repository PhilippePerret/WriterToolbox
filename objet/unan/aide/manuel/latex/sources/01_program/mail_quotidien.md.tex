% <!--
%   CE FICHIER CONTIENT LES DÉFINITIONS GÉNÉRALES DES LIENS
% 
%   DÉFINITION :
% 
%   [un texte identifiant]:  une/path/to/cible  "Le titre optionnel"
% 
%   UTILISATION
% 
%   avec le texte identique :
% 
% un texte identifiant][
% 
%   avec un autre texte :
% 
% autre texte pour le lien][un texte identifiant
% 
%   -->

\section{Mail quotidien}\hypertarget{mail-quotidien}{}\label{mail-quotidien}

… ``ou pas [quotidien]'' devrions-nous ajouter au titre, car ce mail dont nous allons parler ici peut être quotidien seulement si vos \hyperlink{preferences-auteur}{préférences} le déterminent.

\subsection{État des lieux de votre programme}\hypertarget{tat-des-lieux-de-votre-programme}{}\label{tat-des-lieux-de-votre-programme}

Ce mail quotidien vous fait tous les jours un \textbf{état des lieux de votre programme}, vous indiquant vos nouveaux travaux, vos travaux en retard, vos travaux à démarrer ainsi que l'échéance des travaux qui se poursuivent et toute information qui peut vous être utile en temps donné.

\subsection{Mail seulement si nouveaux travaux}\hypertarget{mail-seulement-si-nouveaux-travaux}{}\label{mail-seulement-si-nouveaux-travaux}

Si vous ne désirez pas recevoir ce mail quotidiennement, vous pouvez régler vos \hyperlink{preferences-auteur}{préférences} pour ne le recevoir que lorsque vous avez de nouveaux travaux, donc à un changement de \hyperlink{explicationjourprogrammejourreel}{jour-programme} (\emph{à titre de rappel, la durée exacte de ce jour-programme peut varier en fonction de votre rythme}).

