% <!--
%   CE FICHIER CONTIENT LES DÉFINITIONS GÉNÉRALES DES LIENS
% 
%   DÉFINITION :
% 
%   [un texte identifiant]:  une/path/to/cible  "Le titre optionnel"
% 
%   UTILISATION
% 
%   avec le texte identique :
% 
% un texte identifiant][
% 
%   avec un autre texte :
% 
% autre texte pour le lien][un texte identifiant
% 
%   -->

\section{Rythme de travail}\hypertarget{rythme-travail}{}\label{rythme-travail}

Par défaut, votre programme se règle sur un \emph{rythme moyen}. Ce rythme est très exactement celui qui vous permet d'accomplir le programme en une année très précisément.

Il est néanmoins important de comprendre que ce rythme est celui que doit adopter un auteur qui peut consacrer au moins une à deux heures à son programme quotidiennement. Si vous ne pouvez pas consacrer ce temps, vous aurez à régler ce rythme sur un niveau moindre. Si vous pouvez au contraire consacrer plus de temps à votre écriture sur ce programme, ou que vous le suivez pour une deuxième ou troisième fois sans avoir besoin de relire toutes les pages de cours par exemple, alors vous pourrez augmenter ce rythme (dans ce cas, vous effectuerez le programme en moins d'un an).

