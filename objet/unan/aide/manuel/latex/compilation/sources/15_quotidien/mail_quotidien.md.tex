% <!--
%   CE FICHIER CONTIENT LES DÉFINITIONS GÉNÉRALES DES LIENS
% 
%   DÉFINITION :
% 
%   [un texte identifiant]:  une/path/to/cible  "Le titre optionnel"
% 
%   UTILISATION
% 
%   avec le texte identique :
% 
% un texte identifiant][
% 
%   avec un autre texte :
% 
% autre texte pour le lien][un texte identifiant
% 
%   -->

\section{Mail quotidien}\hypertarget{mail-quotidien}{}\label{mail-quotidien}

… ``ou pas [quotidien]'' devrions-nous ajouter au titre, car ce mail dont nous allons parler n'est quotidien que pour deux raisons~{}:

\begin{itemize}
\item votre [rythme][] est \emph{moyen} (\emph{normal}) et vous changez donc de \hyperlink{explicationjourprogrammejourreel}{jour-programme} tous les jours~{};
\item votre rythme est plus lent que le rythme \emph{normal} mais vous avez réglé vos \hyperlink{preferences-auteur}{préférences} de telle sorte que vous recevez ce message quotidiennement.
\end{itemize}

Dans tous les autres cas, vous ne recevez ce message qu'au moment de votre passage d'un jour-programme à un autre.

\subsection{État des lieux de votre programme}\hypertarget{tat-des-lieux-de-votre-programme}{}\label{tat-des-lieux-de-votre-programme}

Ce mail quotidien —~{}nous l'appellerons ainsi même s'il n'est pas forcément journalier~{}— vous présente un \textbf{état des lieux de votre programme}, vous indiquant vos nouveaux travaux, vos travaux en retard, vos travaux à démarrer ainsi que l'échéance des travaux qui se poursuivent et toute information qui peut vous être utile en temps donné.

Il peut ressembler à ça~{}:

\begin{center}
\image[scale=0.6]{mail/mail-quotidien-1.png}
\image[scale=0.6]{mail/mail-quotidien-2.png}
\end{center}

Il vous informe de votre \hyperlink{explicationjourprogrammejourreel}{jour-programme} courant, de votre \hyperlink{explication-nombre-points}{nombre de points}, des tâches à démarrer, etc.

