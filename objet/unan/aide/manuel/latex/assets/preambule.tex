% Packages utiles
\usepackage[french]{babel}
% Pour demander à frenchb de ne pas s'occuper des listes à
% puce.
\frenchbsetup{StandardLists=true}

\usepackage[utf8]{inputenc}
\usepackage[OT1,T1]{fontenc}
% Pour la fabrication de l'index
\usepackage{makeidx}
% Pour améliorer la mise en page
\usepackage[]{fancyhdr}
\pagestyle{fancy}
\fancyhf{}
\fancyhead{}

\usepackage{ifthen}

\usepackage{hyperref}
\hypersetup{
colorlinks=false, % Pour ne pas mettre les liens en couleur
linkcolor=blue,
}
% Pour les guillemets français
\usepackage[babel=true]{csquotes}
% Pour pouvoir définir la mise en forme des références, par exemple un
% "p." avant le numéro de page.
\usepackage{varioref} % => \vref au lieu de \ref
% Pour pouvoir utiliser la font Zapf dingbats qui permet de
% modifier les puces des listes
\usepackage{pifont}
