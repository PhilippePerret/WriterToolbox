% Définition de la première de couverture

\begin{titlepage}
	% Titre de la série
	\begin{center}\LARGE
		\textit{B.O.A. Éditions}
	\end{center}
	\fontfamily{phv}\selectfont
	\vspace*{\stretch{1}}

	% Titre
	\begin{flushleft}\huge\bfseries
		% Le titre du livre
    LE PROGRAMME \\
    \enquote{UN AN UN SCRIPT} \\
		{\small MANUEL DE L'AUTEUR\fem{E}}
	\end{flushleft}

	% Une ligne séparant l'auteur et le titre
	\hrule

	% Auteur
	\begin{flushright}
		Philippe Perret
	\end{flushright}

	% Sous-titre
	\begin{flushleft}\itshape\small
    % Ici on pourrait ajouter un sous-titre
	\end{flushleft}

	\vspace*{\stretch{2}}
	% Éditeur
	\begin{center}
    version Mai 2016
	\end{center}
\end{titlepage}

% Nécessaire avec \maketitle, mais avec une page propre comme
% ici, est-ce vraiment nécessaire ? Apparemment, non.
% \thispagestyle{fancy}
