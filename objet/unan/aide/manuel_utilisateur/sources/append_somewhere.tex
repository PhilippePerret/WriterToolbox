%
% AJOUTER DANS CE FICHIER TOUS LES TEXTES QUI NE SONT PAS ENCORE PLACÉS
% QUELQUE PART
% 
% Ils seront ajoutés à la fin du document PDF, donc ils peuvent contenir des
% labels utilisés dans le texte

\chapter{À placer quelque part dans le document}
\label{a_placer_quelque_part_dans_le_document}


\subsection{Jour-programme et jour réel}\label{explicationjourprogrammejourreel}

Quand on suit le programme \unan{}, il faut comprendre qu'il y a deux sortes de jours~: les \enquote{jour-programme} et les \enquote{jours réels}. Lorsque l'on suit le programme en rythme \enquote{normal}, c'est-à-dire \enquote{moyen}, un jour-programme et un jour réel sont équivalent ou, pour le dire autrement, un jour-programme dure un jour réel.

Le programme \unan{} étant pensé sur un an, il est décomposé en \textbf{366 jours-programme} qui chacun définissent le programme complet.

Lorsque l'on ralentit son rythme de travail, les jours réels sont plus nombreux —~on peut faire le programme \unan{} en un an et demi par exemple~— mais le nombre de jours-programme ne varie pas, bien entendu. Donc ces jours-programme doivent se \emph{rallonger} pour durer le temps voulu.

Si, par exemple, vous suivez le programme \unan{} avec un rythme deux fois plus lent que le rythme moyen, vous mettrez deux ans pour suivre l'intégralité du programme (\emph{cela peut vous sembler long, en réalité, il faut bien plus de temps que ça pour apprendre la dramaturgie et réussir son premier bon scénario~!}). Deux ans pour suivre le programme prévu sur une année, cela signifie qu'un \emph{jour-programme} durera \textbf{2 jours réels}.
